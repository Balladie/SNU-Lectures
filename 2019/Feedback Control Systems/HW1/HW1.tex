\documentclass{article}

% Call the style package
\usepackage{fun3style}
\usepackage{amsmath}
\usepackage{graphicx}

\graphicspath{ {./images/} }
    
\title{Homework \#1}
\author{GangIn Park}
\date{FCS Fall 2019}

\begin{document}

\maketitle

\subsection*{Problem 1.}
    \begin{enumerate}
    \item
    Note that
    $$ \dot{x_{1}} = x_{2} $$
    $$ \dot{x_{2}} = -\frac{c}{m}\dot{y} - \frac{k}{m}y - \frac{k}{m}a^{2}y^{3} + u. $$
    The state space model is then given by
    $$ \dot{x} = f(x, u) = 
    \left[
    \begin{matrix}
    x_{2} \\
    -\frac{k}{m}(x_{1} + a^{2}{x_{1}}^{3}) - \frac{c}{m}x_{2} + u
    \end{matrix}
    \right]
    .$$
    
    \item
    The Jacobian matrices of $A$ and $B$ for linearization is
    $$ A = \frac{\partial{f}}{\partial{x}} = 
    \left[
    \begin{matrix}
    \frac{\partial{f_1}}{\partial{x_1}} & \frac{\partial{f_1}}{\partial{x_2}} \\
    \frac{\partial{f_2}}{\partial{x_1}} & \frac{\partial{f_2}}{\partial{x_1}}
    \end{matrix}
    \right]
    = 
    \left[
    \begin{matrix}
    0 & 1 \\
    -\frac{k}{m}(1+3a^{2}{x_1}^{2}) & -\frac{c}{m}
    \end{matrix}
    \right]
    ,$$
    $$
    B = \frac{\partial{f}}{\partial{u}} = 
    \left[
    \begin{matrix}
    \frac{\partial f_1}{\partial u_1} \\
    \frac{\partial f_2}{\partial u_1}
    \end{matrix}
    \right]
    =
    \left[
    \begin{matrix}
    0 \\
    1
    \end{matrix}
    \right].
    $$
    Around the origin, let $x_e=0$ and $u_e=0$. Then
    $$ A = \frac{\partial{f}}{\partial{x}}\Bigr|_{x_e}
    = 
    \left[
    \begin{matrix}
    0 & 1 \\
    -\frac{k}{m} & -\frac{c}{m}
    \end{matrix}
    \right], $$
    $$ B = \frac{\partial{f}}{\partial{u}}\Bigr|_{u_e}
     = 
    \left[
    \begin{matrix}
    0 \\
    1
    \end{matrix}
    \right]. $$
    The final linearized system is then given by
    $$ \dot{x} = 
    \left[
    \begin{matrix}
    0 & 1 \\
    -\frac{k}{m} & -\frac{c}{m}
    \end{matrix}
    \right] 
    x + 
    \left[
    \begin{matrix}
    0 \\
    1
    \end{matrix}
    \right] 
    u. $$
    
    \item
    Let
    $$ P = 
    \left[
    \begin{matrix}
    p_{11} & p_{12} \\
    p_{21} & p_{22}
    \end{matrix}
    \right]
    $$
    where $p_{12}=p_{21}$ and $p_{11}, p_{12}, p_{21}, p_{22}$ are nonnegative. $P$ should satisfy the given equation:
    $$ PA+A^{T}P=-I. $$
    This gives the following equations in terms of $p_{11}, p_{12}, p_{22}$:
    $$ -\frac{2k}{m}p_{12}=-1 $$
    $$ p_{11}-\frac{c}{m}p_{12}-\frac{k}{m}p_{22}=0 $$
    $$ 2p_{12}-\frac{2c}{m}p_{22}=-1. $$
    Solving these equations, $P$ is given by
    $$ P=
    \left[
    \begin{matrix}
    \frac{c}{2k}+\frac{m}{2c}+\frac{k}{2c} & \frac{m}{2k} \\
    \frac{m}{2k} & \frac{m^2}{2ck}+\frac{m}{2c} 
    \end{matrix}
    \right]
    $$
    Let the \textit{Lyapunov function} $V$ as
    $$ V=x^{T}Px. $$
    Note that $V$ is positive definite and radially unbounded. Compute the derivative $\dot{V}$:
    $$ \dot{V}=x^T(PA+A^{T}P)x=-x^{T}Ix. $$
    Because $\dot{V}$ is negative definite, the system is \textit{globally asymptotically stable}. \\
    
    \end{enumerate}
    
\subsection*{Problem 2.}
    \begin{enumerate}
    \item
    The state space model is given by
    $$ \dot{x}=\frac{1}{T}x-\frac{k}{T^{2}}u $$
    $$ y=x-\frac{k}{T}u $$
    
    \item
    For unit step response, let $u=S(t)$. The given equation is
    $$ \dot{x}-\frac{1}{T}x=-\frac{k}{T^{2}}. $$
    Solving this equation where $x(0)=0$:
    $$ \frac{\,d}{\,dt}(e^{-\frac{t}{T}}x)=-\frac{k}{T^{2}}e^{-\frac{t}{T}} $$
    \begin{align}
    e^{-\frac{t}{T}}x& = -\frac{k}{T^{2}}\int_{0}^{t} e^{-\frac{\tau}{T}} {\,d\tau} \\ 
    &= \frac{k}{T}(e^{-\frac{t}{T}}-1)
    \end{align} 
    $$ x = \frac{k}{T}(1-e^{\frac{t}{T}}). $$
    The unit step response is
    $$ y = -\frac{k}{T}e^{\frac{t}{T}}. $$
    The plot of the response depends on the sign of $k$ and $T$, as Figure 1. \\
    \begin{figure}
        \includegraphics[totalheight=8cm]{Images/HW1_2_plot.jpg}
        \caption{Step response for Problem 1.}
    \end{figure} 
    
    For impulse response, let $u=\delta(t)$. The given equation becomes
    $$ \dot{x}-\frac{1}{T}x=-\frac{k}{T^{2}}\delta(t). $$
    Solving this equation where $x(0)=0$:
    $$ \frac{\,d}{\,dt}(e^{-\frac{t}{T}}x)=-\frac{k}{T^{2}}\delta(t)e^{-\frac{t}{T}} $$
    \begin{align}
    e^{-\frac{t}{T}}x& = -\frac{k}{T^{2}}\int_{0}^{t} e^{-\frac{\tau}{T}}\delta(\tau) {\,d\tau} \\ 
    &= -\frac{k}{T^{2}}
    \end{align} 
    $$ x = \frac{k}{T^{2}}e^{\frac{t}{T}}. $$
    The impulse response is
    $$ y = -\frac{k}{T^{2}}e^{\frac{t}{T}}-\frac{k}{T}\delta(t). $$
    The plot of this response also depends on the sign of $k$ and $T$, as Figure 2. \\
    \begin{figure}
        \includegraphics[totalheight=8cm]{Images/HW1_2_plot2.jpg}
        \caption{Impulse response for Problem 1.}
    \end{figure} 
    
    \end{enumerate}
    
\subsection*{Problem 3.}
    \begin{enumerate}
    \item
    Note that for
    $$ \det(sI-A)=s^{2}-3s+1=0, $$
    $s=\frac{3 \pm \sqrt{5}}{2}.$ Because all eigenvalues have positive real parts, the system is \textit{unstable}.
    
    \item
    The controllability matrix is given by:
    $$ W_{c}=[B|AB]=
    \left[
    \begin{matrix}
    1 & 1 \\
    0 & 1
    \end{matrix}
    \right] $$
    Since $\rank{(W_{c})}=2$, the system is controllable. \\
    The controllable canonical form matrices $\Tilde{A}$ and $\Tilde{B}$ can be expressed as
    $$ \Tilde{A}=
    \left[
    \begin{matrix}
    -a_{1} & -a_{2} \\
    1 & 0
    \end{matrix}
    \right],
    \Tilde{B}=
    \left[
    \begin{matrix}
    1 \\
    0 
    \end{matrix}
    \right] $$
    The characteristic polynomial of $A$ and $\Tilde{A}$ should be same: that is,
    $$ \det
    \left[
    \begin{matrix}
    s+a_{1} & a_{2} \\
    -1 & s
    \end{matrix}
    \right]
    =s^{2}+a_1s+a_2=s^2-3s+1 $$
    \begin{align}
    a_1&=-3 \\
    a_2&=1 
    \end{align}
    $$ \Tilde{A}= \left[ \begin{matrix}
    3 & -1 \\
    1 & 0
    \end{matrix} \right]. $$
    The controllable matrix for the controllable canonical form is given by
    $$ \Tilde{W}_{c}= \left[ \begin{matrix}
    1 & 3 \\
    0 & 1
    \end{matrix} \right]. $$
    The inverse of the original controllability matrix is
    $$ {W}_{c}= \left[ \begin{matrix}
    1 & -1 \\
    0 & 1
    \end{matrix} \right]. $$
    Hence we get a desired transformation
    $$ T=\Tilde{W}_c{W_c}^{-1}= \left[ \begin{matrix}
    1 & 2 \\
    0 & 1
    \end{matrix} \right]. $$
    
    \item
    Let $u(t)=-Kx(t)+r$ and $K=\left[ \begin{matrix}
    k_1 & k_2
    \end{matrix} \right]$. Then 
    \begin{align}
    \dot{x}&=(A-BK)x+Br \\
    &= \left[ \begin{matrix}
    1-k_1 & 1-k_2 \\
    1 & 2
    \end{matrix} \right] x+
    \left[ \begin{matrix}
    1 \\
    0
    \end{matrix} \right] r.
    \end{align}
    From $\lambda=-2 \pm j1$, the desired characteristic polynomial is given by
    $$ p(s)=s^2+4s+5 $$
    We want to get $K$ so that
    \begin{align}
    \det \left[ \begin{matrix}
    s-1+k_1 & k_2-1 \\
    -1 & s-2
    \end{matrix} \right]
    &=s^2+(k_1-3)s+2k_1+k_2+1 \\
    &=s^2+4s+5.
    \end{align}
    Hence
    $$ K= \left[ \begin{matrix}
    7 & -10
    \end{matrix} \right]. $$
    
    \end{enumerate}
    
\subsection*{Problem 4.}
    \begin{enumerate}
    \item
    Note that
    $$ \dot{x} = \left[ \begin{matrix}
    \frac{v}{L} - \frac{R}{L}i - \frac{K_b}{L}\omega \\
    \frac{K_t}{J}i - \frac{b}{J}\omega \\
    \omega
    \end{matrix} \right]. $$
    Hence the state space model is expressed as
    $$ \dot{x} = Ax+Bu = \left[ \begin{matrix}
    -\frac{R}{L} & -\frac{K_b}{L} & 0 \\
    \frac{K_t}{J} & -\frac{b}{J} & 0 \\
    0 & 1 & 0
    \end{matrix} \right]
    x + \left[ \begin{matrix}
    \frac{1}{L} \\
    0 \\
    0
    \end{matrix} \right]
    v. $$
    
    \item
    With control law, let $v = -Kx+k_fr$ where $K = \left[ \begin{matrix}
    k_1 & k_2 & k_3
    \end{matrix} \right].$
    Then
    $$ A-BK = \left[ \begin{matrix}
    -\frac{k_1+R}{L} & -\frac{k_2}{K_b} & -\frac{k_3}{L} \\
    \frac{K_t}{J} & -\frac{b}{J} & 0 \\
    0 & 1 & 0
    \end{matrix} \right]. $$
    The desired characteristic polynomial for the eigenvalues $\lambda_1 = -1240+1240i$, $\lambda_2 = -1240-1240i$, $\lambda_3 = -1000$ is given by
    \begin{equation}
    (s-\lambda_1)(s-\lambda_2)(s-\lambda_3) 
    \end{equation}
    And consider the characteristic polynomial of $A-BK$:
    \begin{align}
    \det(sI-(A-BK)) &= \det \left[ \begin{matrix}
    s+\frac{R+k_1}{L} & \frac{K_b+k_2}{L} & \frac{k_3}{L} \\
    -\frac{K_t}{J} & s+\frac{b}{J} & 0 \\
    0 & -1 & s
    \end{matrix} \right] \\
    &= s^3+(\frac{b}{J}+{R+k_1}{L})s^2+(\frac{b(R+k_1)}{JL}+\frac{K_t(K_b+k_2)}{JL})s+\frac{K_tk_3}{JL}
    \end{align}
    Comparing eq.(13) with eq.(11), we finally get
    $$ K = \left[ \begin{matrix}
    -3.9904 & -0.0256 & 0.9964
    \end{matrix} \right]. $$
    
    \end{enumerate}
    
\subsection*{Problem 5.}
    \begin{enumerate}
    \item
    
    
    \end{enumerate}

\end{document}